\documentclass[ngerman,addpoints,10pt]{exam}
\usepackage[T1]{fontenc}
\usepackage[utf8]{inputenc}
\usepackage[a4paper]{geometry}
\geometry{verbose,tmargin=2.5cm,bmargin=2.5cm,lmargin=2.5cm,rmargin=2.5cm}
\pagestyle{headings}
\setcounter{secnumdepth}{2}
\setcounter{tocdepth}{2}
\usepackage{babel}	
\usepackage{amssymb}
\usepackage{color}

\pagestyle{headandfoot}
\firstpageheader{
\large\bfseries \Large Einführung in die Informatik II - 
Keine Klausur, 8. Mai 2014}{}{}


\firstpagefooter{Dozent: Axel Dürkop}{}{Bitte wenden!}
\runningheader{\large\bfseries Einführung in die Informatik II -
Keine Klausur, 8. Mai 2014}{}{}
\runningfooter{Dozent: Axel Dürkop}{}{Seite \thepage} 

\pointsdroppedatright
\extrawidth{-1cm}

%\printanswers
%\CorrectChoiceEmphasis{\color{green}\bfseries}

\renewcommand{\solutiontitle}{\noindent\textbf{Lösung:}\par\noindent}




\begin{document}

	\fullwidth{\noindent \Large Thema: Client-/Server-Kommunikation, HTML und CSS}
		
\begin{flushleft}
		Anzahl der Fragen: \textbf{\numquestions}\\
		Anzahl der erreichbaren Punkte: \textbf{\numpoints}\\
		Dauer der Prüfung: \textbf{30 Minuten}
\end{flushleft}


		
	\begin{questions}	
	
		\fullwidth{\Large \textbf{Fragen zu HTML und CSS}}

		\question[10]
		\textbf{Wofür steht die Abkürzung HTML?}
		\droppoints
		
		\begin{checkboxes}
			\choice Hypertext Makeup Language
			\choice Helpertext Markup Language
			\choice Hyperterminal Markup Language
			\choice Hypertags Markup Language
			\CorrectChoice Hypertext Markup Language
		\end{checkboxes}
		\bigskip

		% --------------------------------------------------

		\question[10]
		\textbf{Welche Attribute sind erforderlich für die W3C-gemäße Verwendung des \texttt{<img/>}-Elements (Mehrfachantwort)?}
		\droppoints
		
		\checkboxchar{$\Box$}
		\checkedchar{$\blacksquare$}

		\begin{checkboxes}
			\CorrectChoice src
			\choice href
			\CorrectChoice alt
			\choice rel
			\choice title
		\end{checkboxes}
		\bigskip
		
		% --------------------------------------------------
						
		\question[10]
		\textbf{Erläutern Sie mit wenigen Worten den Unterschied zwischen HTML und CSS!}
		\droppoints
		\begin{solutionorbox}[4cm]
			\begin{flushleft}
			HTML dient dazu, Informationen eines Webdokuments gemäß ihrer Bedeutung auszuzeichnen.
			CSS wird verwendet, um die Informationen der Seite zu formatieren.
			\end{flushleft}
		\end{solutionorbox}
		\bigskip
		
		% --------------------------------------------------
						
		\question[10]
		\textbf{Erläutern Sie den Unterschied zwischen Klassen und IDs im Bezug auf CSS!}
		\droppoints
		\begin{solutionorbox}[4cm]
			\begin{flushleft}
			IDs werden verwendet, um ein HTML-Element im Dokument eindeutig auszuzeichnen.
			Die mehrmalige Verwendung der ID im selben Dokument ist nicht erlaubt. 
			CSS-Klassen werden im Sinne von Stilvorlagen verwendet und können mehrfach für
			Elemente vergeben werden, um diesen allen das gleiche Aussehen zu geben. Eine
			Änderung in der Klasse wirkt sich folglich auf alle Instanzen aus.
			\end{flushleft}
		\end{solutionorbox}
		\bigskip
		
		% --------------------------------------------------	
		\newpage

		\fullwidth{\Large \textbf{Fragen zu Client und Webserver}}			
				
		\question[10]
		\textbf{Aus welcher Folge von Ereignissen bestehen Aufruf und Auslieferung der Webseite \textit{www.heise.de}?}
		\droppoints
		\begin{solutionorbox}[6cm]
			\begin{enumerate}
			\item Eingabe der URL im Browser
			\item Abfrage der IP-Adresse des Zielservers beim Domain Name Service (DNS)
			\item Anfrage (Request) des gewünschten Dokuments beim Zielservers unter der erhaltenen IP
			\item Antwort des Webservers und Auslieferung des gewünschten Dokuments sofern vorhanden
			\item Anzeige des Dokuments im Browser
			\end{enumerate}
		\end{solutionorbox}
		\bigskip
		
		% --------------------------------------------------
		
		\question[10]
		\textbf{Erklären Sie das Konzept des Wurzelverzeichnis eines Webservers!}
		\droppoints
		\begin{solutionorbox}[6cm]
			\begin{flushleft}
			Das Wurzelverzeichnis ist der Startpunkt der Verzeichnisstruktur eines Webservers. 
			Unterhalb des Wurzelverzeichnisses liegen die Dokumente, die der Webserver ausliefern kann. 
			Der Dokumentenpfad wird beim Request an die URL angehängt, wobei der \texttt{/} hinter der IP 
			oder Domain dem Wurzelverzeichnis entspricht.\\
			Das Wurzelverzeichnis des Serverbetriebssystems ist nicht zu verwechseln mit dem \\
			Wurzelverzeichnis
			des Webservers!
			\end{flushleft}
		\end{solutionorbox}
		\bigskip

	\end{questions}

	\hqword{Frage}
	\hpword{Punktzahl}
	\hsword{Davon erreicht}
	\htword{\textbf{Summe}}


	\fullwidth{\Large \textbf{Bewertung}}			
	\fullwidth{\gradetable[h]}

	\bigskip
	\bigskip

	\fullwidth{
		\fbox{\parbox{14cm}{
		
		\begin{flushleft}
		\noindent\textbf{\Large Hinweis}\par\noindent Diese Klausur ist keine Klausur, auch wenn Sie jetzt den Eindruck haben.
		Sie beantworten die Fragen jetzt oder außerhalb der Veranstaltung und
		prüfen eigenverantwortlich Ihren Kenntnisstand. Anschließend behalten Sie das Arbeitsblatt und
		verbergen Ihre Antworten vor Ihrem Dozenten.\\

		\end{flushleft}
		}}
	}
\end{document}
